\chapter{Demo}

\section{Document Structure}

\subsection{Folders}
\begin{mytable}
    \begin{tabular}{ll}
        \textbf{Name} & \textbf{Comment} \\
        \hline
        \texttt{notes}       & folder storing notes \\
        \texttt{figures}     & folder storing regular figures \\
        \texttt{fonts}       & folder storing fonts used in notes \\
        \texttt{style}       & folder storing format control files \\
        \texttt{tikz}        & folder storing TikZ figure source code
    \end{tabular}
\end{mytable}



\subsection{Format Control Files}
\begin{mytable}
    \begin{tabular}{ll}
        \textbf{Name} & \textbf{Comment} \\
        \hline
        \texttt{mycolor.sty} & document defining colors \\
        \texttt{myenv.sty}   & document defining environments \\
        \texttt{myfont.sty}  & document defining fonts \\
        \texttt{mypage.sty}  & document defining page format \\
        \texttt{note.cls}    & document defining note class
    \end{tabular}
\end{mytable}


\section{Example}
\begin{term}[\hypertarget{ref_demo}{Euler's Formula}]
    \textbf{Euler's formula} is given as
    \begin{equation}
        \definition{\cos\theta + j\sin\theta = e^{j\theta}}
        \label{def_demo}
    \end{equation}
    Where
    \begin{alignat*}{2}
        &j\ &&\text{--- Imaginary unit}
    \end{alignat*}

    Since
    \[ e^{-j\theta} = \cos(-\theta) + j\sin(-\theta) = \cos\theta - j\sin\theta \]
    
    Hence
    \begin{equation}
        \corollary{\cos\theta = \frac{e^{\, j\theta} + e^{-j\theta}}{2}}
        \label{cor_demo}
    \end{equation}
\end{term}


\begin{example}
    % example goes here
    \hyperlink{ref_demo}{Euler's formula} is described by equation \ref{def_demo}.
\end{example}


\begin{solution}
    % solution goes here
\end{solution}
